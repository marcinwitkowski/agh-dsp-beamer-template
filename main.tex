\documentclass[aspectratio=169]{beamer}
\usepackage[utf8]{inputenc}

\usepackage{tabularx}

\usetheme{Madrid}
\usecolortheme{beaver}
\graphicspath{{images/}}


%------------------------------------------------------------
%This block of code defines the information to appear in the
%Title page
\title[Short Title] %optional
{Rulling the world with signal processing techniqes}

\author[M. Witkowski, M. Rybicka, K. Kowalczyk] % (optional)
{\\Marcin Witkowski \and Magdalena Rybicka \and Konrad Kowalczyk}

\institute[AGH] % (optional)
{\\\\
\begin{tabular}[h]{ccc}
    \includegraphics[height=1.4cm]{logo_DSP_wavelet_AGH.png}\;\;\;\;\;\;\;\;& 
      \includegraphics[height=2.2cm]{agh_nzw_s_en_1w_wbr_cmyk.eps}\;\;\;\;\;\;\;\;&  
      \includegraphics[height=1.8cm]{conf.png}%

  \end{tabular} \\\\

}

\date[EUSIPCO 2021] % (optional)
{ASMSP-P6: Speech Enhancement and Spoken Document Retrieval\\
Thursday, 26 August 2021, 13:30 - 16:30 IST (UTC +01:00)\\}

\logo{ \includegraphics[height=0.8cm]{logo_DSP_wavelet_AGH_short.png}
       \includegraphics[height=0.8cm]{agh_znk_wbr_cmyk.eps}
     }


%End of title page configuration block
%------------------------------------------------------------



%------------------------------------------------------------
%The next block of commands puts the table of contents at the 
%beginning of each section and highlights the current section:

\AtBeginSection[]
{
  \begin{frame}
    \frametitle{Table of Contents}
    \tableofcontents[currentsection]
  \end{frame}
}
%------------------------------------------------------------


\begin{document}

%The next statement creates the title page.
\frame[plain]{\titlepage}
%\frame{\titlepage}


%---------------------------------------------------------
%Changing visivility of the text
\begin{frame}
\frametitle{State-of-the-art}
This is a text in second frame. For the sake of showing an example.

\begin{itemize}
    \item<1-> Text visible on slide 1
    \item<2-> Text visible on slide 2
    \item<3> Text visible on slides 3
    \item<4-> Text visible on slide 4
\end{itemize}
\end{frame}

%---------------------------------------------------------


%---------------------------------------------------------
%Example of the \pause command
\begin{frame}
\frametitle{Method}
In this slide \pause

the text will be partially visible \pause

And finally everything will be there
\end{frame}
%---------------------------------------------------------


%---------------------------------------------------------
%Highlighting text
\begin{frame}
\frametitle{Blocks}

In this slide, some important text will be
\alert{highlighted} because it's important.
Please, don't abuse it.

\begin{block}{Remark}
Sample text
\end{block}

\begin{alertblock}{Important theorem}
Sample text in red box
\end{alertblock}

\begin{examples}
Sample text in green box. The title of the block is ``Examples".
\end{examples}
\end{frame}
%---------------------------------------------------------
%---------------------------------------------------------
%Two columns
\begin{frame}
\frametitle{Systems}
\centering
image
\end{frame}
%-------------------------------------------------------

%---------------------------------------------------------
%Two columns
\begin{frame}
\frametitle{Results}

\begin{columns}

\column{0.5\textwidth}
This is a text in first column.
$$E=mc^2$$
\begin{itemize}
\item First item
\item Second item
\end{itemize}

\column{0.5\textwidth}
This text will be in the second column
and on a second tought this is a nice looking
layout in some cases.
\end{columns}
\end{frame}
%-------------------------------------------------------
%---------------------------------------------------------
%Example of the \pause command
\begin{frame}
\frametitle{Want more?}

\centering
Meet me at \\

Session: ASMSP-P6: Speech Enhancement and Spoken Document Retrieval\\
Session Time: Thursday, 26 August, 13:30 - 16:30 IST (UTC +01:00)\\
or\\
witkow@agh.edu.pl

\end{frame}
%---------------------------------------------------------

\end{document}